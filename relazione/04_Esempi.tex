
\section{Dati d'esempio}
	\subsection{set1}

	\subsubsection{File .dat}
		\lstinputlisting[language={}]{../atlas_01.dat}

\newpage



\begin{figure}[h]
\caption{Example of a parametric plot ($\sin (x), \cos(x), x$)}
% \centering

	\includegraphics[width=4cm]{results01}
	\hspace{1cm}
	\includegraphics[width=4cm]{results02}
\end{figure}


\begin{figure}[h]
\caption{Example of a parametric plot ($\sin (x), \cos(x), x$)}
% \centering
\begin{center}
	\includegraphics[width=4cm]{results03}
	\hspace{1cm}
	\includegraphics[width=4cm]{results04}
\end{center}
\end{figure}


\begin{figure}[h]
\caption{Example of a parametric plot ($\sin (x), \cos(x), x$)}
% \centering

	\includegraphics[width=4cm]{results05}
	\hspace{1cm}
	\includegraphics[width=4cm]{results06}
\end{figure}


	\newpage





6) Extra


funzionalità aggiuntive


soluz.txt
lo script .run crea un file di nome soluz.txt che verrà poi usato da makeImg.


makeImg
è stato sviluppato un piccolo programma javascript di supporto per la creazione di una immagine apposita che permetta una migliore visualizzazione della soluzione di ogni set di dati.