
%	\maketitle
\begin{center}
	\hspace{0pt}
	\vfill
	\Huge{
		\textbf{\thetitle} \\
		\LARGE{\textbf{Progetto: texture atlas}} \\
		\ \newline \newline \newline
		\Large{	Studente: \textbf{\theauthor} } \\
		\normalsize{{Matricola: \textbf{\matNum}} }
	}
	\vfill
	\hspace{0pt}
	%\vspace*{\fill}
	%\pagebreak
\end{center}
\newpage

\tableofcontents    
\newpage


\iffalse


%\begin{center}
%	\hspace{0pt}
%	\vfill
	\section{Abstract} 
	\ \newline

Il problema trattato è quello delle texture atlas, ovvero immagini di dimensioni notevoli utilizzate come raccoglitori di altre immagini, utili per accelerare l'accesso (spesso da parte della GPU) alle texture utilizzate. \\
Si vuole trovare la minima dimensione D di un'immagine quadrata tale che in un immagine D x D sia possibile inserire tutte le texture date.
	delle texture atlas

%	\vfill
%	\hspace{0pt}
%\end{center}


\newpage

\fi




  % descrizione + caratteristiche (?)

	\section{Descrizione del problema}


 
%Il problema in esame richiede di trovare la minima lunghezza D per cui un'immagine D*D contenga tutte le immagini facenti parte della texture atlas. \\
%vincolo principale del problema è che D deve poter contenere tutte le immagini, quindi D dev'essere almeno uguale alla posizione+dimensione (W, H) di ogni immagine. \\
%altro vincolo comporta la non sovraposizione per ogni coppia di immagini dell'insieme. \\

Il problema trattato è quello delle texture atlas, ovvero immagini di dimensioni spesso notevoli utilizzate come raccoglitori di altre immagini, utili per accelerare l’accesso (soprattutto da parte della GPU) alle texture utilizzate. \\
Si vuole trovare la minima lunghezza D tale che un’immagine quadrata di dimensioni DxD contenga tutte le texture comprese in un set dato. \\
Il vincolo principale del problema è che D deve poter contenere tutte le immagini, quindi D dev’essere almeno uguale alla posizione+dimensione (in ascissa e ordinata) di ogni immagine. \\
Il secondo vincolo principale comporta invece la non sovrapposizione di due texture. Questo vincolo è presente per ogni coppia di immagini diverse dell’insieme. \\
\ \\
Altri vincoli secondari riguardano la possibilità di permettere rotazioni delle texture, tener conto di un piccolo bordo per ogni immagine e limitare i valori assegnabili a D alle sole potenze di 2.







	\newpage