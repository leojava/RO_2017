

  %descrizione modello + spiegazione (vincoli etc)

	\section{Modello matematico}


\begin{align*}	% l'asterisco toglie il numero alle righe (: equazioni)
minimize \ \ D       \ \ \ \ \   & \\	% & per allineare
%
0 \leq X_i           \ \ \ \ \   &\forall{ i \in I} \\
X_i +W_i \leq D      \ \ \ \ \   &\forall{ i \in I} \\
0 \leq Y_i           \ \ \ \ \   &\forall{ i \in I} \\
Y_i +H_i \leq D      \ \ \ \ \   &\forall{ i \in I} \\ \\
%
%
Cx_{i,j} = X_i - (X_j + W_j)       \ \ \ \ \   &\forall{ i \in I, j \in J: i \neq j} \\
Cy_{i,j} = Y_i - (Y_j + H_j)       \ \ \ \ \   &\forall{ i \in I, j \in J: i \neq j} \\ \\
%
%
Cx_{i,j} \leq M *befX_{i,j} -1           \ \ \ \ \   &\forall{ i \in I, j \in J: i \neq j} \\
Cx_{i,j} \geq 0 + M*(1-befX_{i,j})       \ \ \ \ \   &\forall{ i \in I, j \in J: i \neq j} \\ \\
%
%
Cy_{i,j} \leq M *befY_{i,j} -1           \ \ \ \ \   &\forall{ i \in I, j \in J: i \neq j} \\
Cy_{i,j} \geq 0 + M*(1-befY_{i,j})       \ \ \ \ \   &\forall{ i \in I, j \in J: i \neq j} \\ \\
%
%
(1-befX_{i,j})+(1-befX_{j,i})+ \ \ \ \ & \\ (1-befY_{i,j})+(1-befY_{j,i}) \leq 3       \ \ \ \ \   &\forall{ i \in I, j \in J: i \neq j} \\ \\
%
%s.t. l1{i in I, j in I:i!=j}: (Cx[i,j]) <= bigM*befX[i,j]+-1                 ;			#problema con una unità?
%s.t. l2{i in I, j in I:i!=j}: (Cx[i,j]) >= 0             -bigM*(1-befX[i,j]);		#problema con una unità?
%
%
\\
X_i \in \NaturalSet      \ \ \ \ &\forall{i \in I} \\
Y_i \in \NaturalSet      \ \ \ \ &\forall{i \in I} \\
Cx_{i,j} \in \IntegerSet \ \ \ \ &\forall{i \in I,j \in I} \\
Cy_{i,j} \in \IntegerSet \ \ \ \ &\forall{i \in I,j \in I} \\
BefX{i,j} \in \BinSet    \ \ \ \ &\forall{i \in I,j \in I} \\
BefY{i,j} \in \BinSet    \ \ \ \ &\forall{i \in I,j \in I} 
%
 \end{align*}


\iffalse
		\subsection{Variabili Decisionali}

		Bef{x,y}
		\subsection{Parametri}
		W
		H
		useBleding
		usePowTwo
		useRotation
		\subsection{Vincoli e funzione obiettivo}
		min D

		Cx Cy
		\newpage
\fi



\iffalse
Changelog del modello


Il modello è passato per più fasi incrementali:
* La prima versione risolve il problema base
* Quindi è stato aggiunto il supporto per il bleeding
* La terza versione ha aggiunto la possibilità di ruotare le immagini
* La quarta e ultima versione permette di fissare la dimensione della texture atlas a potenze di 2
\fi





\subsection{Insiemi}
\subsubsection{I}
I è l'insieme delle immagini da posizionare nel texture atlas.

\subsubsection{P}
P è l'insieme delle potenze di 2;


\renewcommand{\Xi}{X\textsubscript{i}}
\newcommand{\Yi}{Y\textsubscript{i}}
\newcommand{\Wi}{W\textsubscript{i}}
\newcommand{\Hi}{H\textsubscript{i}}
\newcommand{\beforeXij}{beforeX\textsubscript{i,j}}
\newcommand{\beforeYij}{beforeY\textsubscript{i,j}}
\newcommand{\ri}{r\textsubscript{i}}

\newcommand{\isTp}{is2\textsubscript{p}}

\subsection{Variabili Decisionali}

\subsubsection{D}
D è la variabile intera positiva che indica la dimensione del texture Atlas

\subsubsection{X e Y}
\Xi{} e \Yi{} sono le variabili intere non negative che definiscono la posizione dell'immagine i all'interno del texture atlas.

\subsubsection{W e H}
\Wi{} e \Hi{} sono le variabili intere positive che indicano rispettivamente la larghezza e l'altezza finali occupate dall'immagine i.


\subsubsection{BeforeX e BeforeY}

{\beforeXij} è la variabile binaria che indica se, sull'asse delle ascisse, l'immagine i finisce prima che l'immagine j inizi 
%(in caso di verità: true)
. Se sia \beforeXij{} che beforeX\textsubscript{j,i} sono a 1  allora le proiezioni sulle ascisse delle texture i e j si sovrappongono. \\
{\beforeYij} è lo stesso tipo di variabile, ma per l'asse delle ordinate.

\iffalse
/*
BeforeY
 Questo vincolo esiste sia per l'asse x che per l'asse y e si ripete per ogni coppia ordinata di texture. 
*/
\fi

\subsubsection{r}

{\ri} è la variabile binaria che indica se l'immagine i è ruotata di 90\degree\ (\ri{}=1) o non ruotata (\ri{}=0)

\subsubsection{is2}
\isTp{} variabile binaria che indica se D è uguale a una specifica potenza di 2.







\newcommand{\widthi}{width\textsubscript{i}}
\newcommand{\heighti}{height\textsubscript{i}}


\newcommand{\footBleeding}{L'effetto bleeding si può manifestare nei programmi che sfruttano l'accelerazione hardware. In essi i pixel vengono presi con delle coordinate reali in [0,1] e i pixel subito fuori dall'immagine possono "sporcare" quelli di contorno. Perciò si ricopiano questi ultimi in un bordo di 1 pixel attorno a tutta l'immagine.}
\newcommand{\footRotation}{Spiegare differenza opengl et sim. Verso solo sw.}
\newcommand{\footTwoPowers}{Alcune librerie spesso esigono immagini quadrate il cui lato dev'essere una potenze di 2.}



\subsection{Parametri}

\subsubsection{width e height}

\widthi{} e \heighti{} sono i parametri naturali che indicano le dimensioni originali delle texture.

\subsubsection{bigM}
bigM è un parametro naturale grande a piacere, di default a 100’000’000. Indica l'infinito.

\subsubsection{useNoBleeding}
useNoBleeding è un parametro binario, di default a 0, che indica se tenere conto dell'effetto bleeding\footnote{\footBleeding} nelle texture atlas. 

\subsubsection{allowRotation}
allowRotation è un parametro binario, di default a 0, che indica se permettere che le immagini vengano posizionate ruotate\footnote{\footRotation} di 90\degree.%.rotazioni alle texture (ovvero se scambiare w e h).



\subsubsection{usePowerOfTwo}
usePowerOfTwo è un parametro binario, di default a 0, che indica se limitare i possibili valori di D alle sole potenze di 2\footnote{\footTwoPowers}.

\subsubsection{powersOf2}
powersOf2 : set di potenze di 2. Serve perché altrimenti servirebbe un vincolo non lineare su D per limitarne i valori alle sole potenze di 2. \\














\subsection{funzione obiettivo}
minimizzaLato min D è la funzione obiettivo, dove si cerca di minimizzare D.




\subsection{vincoli}

\subsubsection{minimaX e minimaY}
minimaX si assicura che per ogni immagine, D la possa contenere sulle ascisse. \\
minimaY è lo stesso vincolo sulle ordinate.

\subsubsection{beforeXU, beforeXL e beforeYU, beforeYL}
beforeXU e beforeXL sono i vincoli che si assicurano che beforeX (i,j) abbia il corretto valore. Essi pongono un limite superiore e un limite inferiore all'espressione xi-xj-wj (=ci,j) e risultano in beforeX i,j=1 se ci,j appartiene a [0,bigM], beforeX i,j=0 se ci,j appartiene a [-bigM, -1].
differenza tra l'inizio dell'immagine i e  la fine dell'immagine j.






/*


Cx variabile d'aiuto per befX; dicono quanto spazio c'è a “sinistra” tra l'inizio di i e la fine di j
* >= 0 intersezione non possibile, 
* <0 intersezione possibile
Cy simile per befY
*/




beforeYU e beforeYL sono i corrispettivi vincoli per la variabile beforeY.


\subsubsection{noIntersezioni}
noIntersezioni si assicura che non ci siano intersezioni, ovvero sovrapposizioni sia sulle proiezioni sulle ascisse che sulle ordinate. Esso limita la somma di beforeX e beforeY per ogni coppia non ordinata di immagini (i,j) a 3, ovvero ci può essere al massimo una sovrapposizione su un asse tra ascisse e ma non su entrambe (ovvero un'intersezione).



\subsubsection{rotazione}
rotazione si assicura che le immagini possano ruotare solo se il relativo parametro allowRotations è impostato a 1.





\subsubsection{larghezza e altezza}
larghezza si assicura di tener conto del bleeding e della possibilità che le immagini possano essere ruotate
altezza è un vincolo simile per l'altezza. \\


\subsubsection{powersOf2U e powersOf2L}
pU e pL sono vincoli che definiscono upper e lower bound per D

\subsubsection{TwoPower}
TwoPower si assicura che, in caso usePowerOfTwo sia impostato a 1, D sia una potenza di 2 tra quelle presenti in powersOf2 



