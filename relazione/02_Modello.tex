

%%%%%%%%%%%%%%%%%%%%
% MODELLO
%%%%%%%%%%%%%%%%%%%%

  %descrizione modello + spiegazione (vincoli etc)

	\section{Modello matematico}

\newcommand{\fori}{\forall{ i \in I }}
\newcommand{\forij}{\forall{ i,j \in I, i \neq j }}


\renewcommand{\Xi}{X\textsubscript{i}}
\newcommand{\Yi}{Y\textsubscript{i}}
\newcommand{\Wi}{W\textsubscript{i}}
\newcommand{\Hi}{H\textsubscript{i}}
\newcommand{\beforeXij}{beforeX\textsubscript{i,j}}
\newcommand{\beforeYij}{beforeY\textsubscript{i,j}}
\newcommand{\ri}{r\textsubscript{i}}


\newcommand{\Xj}{X\textsubscript{j}}
\newcommand{\Wj}{W\textsubscript{j}}

\newcommand{\isTp}{is2\textsubscript{p}}


\newcommand{\widthi}{width\textsubscript{i}}
\newcommand{\heighti}{height\textsubscript{i}}
\newcommand{\powersOfTp}{powersOf2\textsubscript{p}}

\newcommand{\eqname}[1]{#1}


\begin{align*}	% l'asterisco toglie il numero alle righe (: equazioni)
%
%\newcommand{\eqnn}[2]{\label{ #1 } \eqname{ #2 }}
%
minimize \ \ D       \ \ \ \ \   &   & \eqname{(minimizzaLato)} \\	% & per allineare
%
0 \leq X_i           \ \ \ \ \   & \fori{} \\
X_i +W_i \leq D      \ \ \ \ \   & \fori{} & \eqname{(minimaX)} \\ %\hyperref[label]{text} \\
0 \leq Y_i           \ \ \ \ \   & \fori{} \\
Y_i +H_i \leq D      \ \ \ \ \   & \fori{} & \eqname{(minimaY)} \\
\\
%
%
X_i - (X_j + W_j) \leq (\text{bigM}+1) *beforeX_{i,j} -1           \ \ \ \ \   &\forij{} &\eqname{(beforeXU)} \\
X_i - (X_j + W_j) \geq 0 + \text{bigM}*(1-beforeX_{i,j})       \ \ \ \ \   &\forij{} &\eqname{(beforeXL)} \\ \\
%
%
Y_i - (Y_j + H_j) \leq (\text{bigM}+1)*beforeY_{i,j} -1           \ \ \ \ \   &\forij{} &\eqname{(beforeYU)} \\
Y_i - (Y_j + H_j) \geq 0 + \text{bigM}*(1-beforeY_{i,j})       \ \ \ \ \   &\forij{} &\eqname{(beforeYL)} \\ \\
%
%
(1-beforeX_{i,j})+(1-beforeX_{j,i})+ \ \ \ \ & \\
(1-beforeY_{i,j})+(1-beforeY_{j,i}) \leq 3       \ \ \ \ \   & \forij{} & \eqname{(noIntersezioni)} \\ \\
%
%
r_i \leq \text{allowRotations} \ \ \ \ & \fori{} &\eqname{(rotazione)} \\
%
W_i = (\text{\widthi}*(1-r_i) + \text{\heighti}*r_i)+ \ \ \ \ & &\\
(2*\text{useNoBleeding}) \ \ \ \ & \fori{} &\eqname{(larghezza)} \\
%
H_i = (\text{\heighti}*(1-r_i) + \text{\widthi}*r_i)+ \ \ \ \ & &\\
(2*\text{useNoBleeding}) \ \ \ \ & \fori{} &\eqname{(altezza)} \\
%
%
\\
D-\text{powersOf2\textsubscript{p}} \leq \text{bigM}*(1-is2_p)       \ \ \ \ \   & \forij{} & \eqname{(powersOf2U)} \\
D-\text{powersOf2\textsubscript{p}} \geq -\text{bigM}*(1-is2_p)       \ \ \ \ \   & \forij{} & \eqname{(powersOf2L)} \\
\sum_{p \in P} is2_p \geq \text{usePowerOfTwo}       \ \ \ \ \   & & \eqname{(TwoPower)} \\
\\
%s.t. powersOf2U{p in P}:  D-powersOf2[p]<= bigM*(1-is2[p]);
%s.t. powersOf2L{p in P}:  D-powersOf2[p]>= -bigM*(1-is2[p]);
%s.t. TwoPower:  sum{p in P} is2[p]>=usePowerOfTwo;
%
%
%
%
%
D \in \IntegerSet \ \ \ \ & & \\
X_i, Y_i, W_i, H_i \in \IntegerSet      \ \ \ \ & \fori{} \\
beforeX_{i,j}, beforeY_{i,j} \in \BinSet    \ \ \ \ & \forall{i \in I,j \in I} \\
%
r_i \in \BinSet \ \ \ \ & \fori{} \\
%
is2_p \in \BinSet \ \ \ \ & \forall{p \in P} \\
%
%
%
 \end{align*}
NB: i parametri sono scritti in testo normale, mentre le variabili e i nomi dei vincoli e delle funzione obiettivo sono scritti in corsivo.




\iffalse
Changelog del modello

Il modello è passato per più fasi incrementali:
* La prima versione risolve il problema base
* Quindi è stato aggiunto il supporto per il bleeding
* La terza versione ha aggiunto la possibilità di ruotare le immagini
* La quarta e ultima versione permette di fissare la dimensione della texture atlas a potenze di 2
\fi





\subsection{Insiemi}
\subsubsection{I}
I è l'insieme delle immagini da posizionare nel texture atlas.

\subsubsection{P}
P è l'insieme delle potenze di 2.





\newcommand{\footBleeding}{L'effetto bleeding si può manifestare nei programmi che sfruttano l'accelerazione hardware. In essi i pixel vengono presi con delle coordinate reali in [0,1] e i pixel subito fuori dall'immagine possono "sporcare" quelli di contorno. Perciò si ricopiano questi ultimi in un bordo di 1 pixel attorno a tutta l'immagine.}
\newcommand{\footRotation}{Alcuni programmi (soprattutto quelli che sfruttano l'accelerazione hardware) possono renderizzare le texture in modo giusto ache se sull'immagine originaria sono ruotate, mentre altri potrebbero avere problemi.}
\newcommand{\footTwoPowers}{Alcune librerie spesso esigono immagini quadrate il cui lato dev'essere una potenza di 2.}

\newcommand{\footFreedom}{L'utente in verità ha la libertà di usare queste variabili per mettere valori a piacere, non necessariamente potenze di 2.}

\newcommand{\footIsTSum}{Essendo variabili di uguaglianza su una stessa variabile D, questa somma può essere al massimo 1.}




%% VARIABILI

\subsection{Variabili Decisionali}

\subsubsection{D}
D è la variabile intera positiva che indica la dimensione del texture Atlas.

\subsubsection{X e Y}
\Xi{} e \Yi{} sono le variabili intere non negative che definiscono la posizione dell'immagine i all'interno del texture atlas.
% Xi può avere valori in 0..D-Wi

\subsubsection{W e H}
\Wi{} e \Hi{} sono le variabili intere positive che indicano rispettivamente la larghezza e l'altezza finali occupate dall'immagine i.


\subsubsection{BeforeX e BeforeY}


% @TODO
%\hbox{\pdfliteral{0 0 1 rg}\vrule height1cm width8cm depth0cm\pdfliteral{0 g}}

\beforeXij{} è la variabile binaria che indica se, sull'asse delle ascisse, l'immagine j finisce prima che l'immagine i inizi. Se sia \beforeXij{} che beforeX\textsubscript{j,i} sono a 0 allora le proiezioni sulle ascisse delle texture i e j si sovrappongono almeno in parte. \\
Questo vincolo si ripete per ogni coppia ordinata di texture diverse. \\
Per ogni i,j , i possibili valori delle coppie (\beforeXij, beforeX\textsubscript{j,i}) possono essere:
\begin{itemize}
	\itemsep0em
	\item (1,0) se j finisce prima di i
	\item (0,0) se j non finisce prima di i e i non finisce prima di j (quindi c'è sovrapposizione sulle ascisse)
	\item (0,1) se i finisce prima di j
	\item (1,1) è impossibile (poiché dovrebbe essere in contemporanea che j finisca prima di i e i finisca prima di j).
\end{itemize}
\beforeYij{} è la stessa variabile, ma per l'asse delle ordinate.


\subsubsection{r}

{\ri} è la variabile binaria che indica se l'immagine i è ruotata di 90\degree\ (ovvero \ri{} vale 1).

\subsubsection{is2}
\isTp{} è la variabile binaria che indica se D è uguale alla p-esima potenza di 2 dell'insieme P.







%% PARAMETRI

\subsection{Parametri}

\subsubsection{width e height}

\widthi{} e \heighti{} sono i parametri interi positivi che indicano rispettivamente la larghezza e l'altezza originali di ogni immagine i.

\subsubsection{bigM}
bigM è un parametro positivo grande a piacere, di default a 100’000’000. %Indica l'infinito.

\subsubsection{useNoBleeding}
useNoBleeding è un parametro binario, di default a 0, che indica se tenere conto dell'effetto bleeding\footnote{\footBleeding} nel texture atlas. 

\subsubsection{allowRotation}
allowRotation è un parametro binario, di default a 0, che indica se permettere che le immagini vengano posizionate ruotate\footnote{\footRotation} di 90\degree.%.rotazioni alle texture (ovvero se scambiare w e h).


\subsubsection{usePowerOfTwo}
usePowerOfTwo è un parametro binario, di default a 0, che indica se limitare i possibili valori di D alle sole potenze di 2\footnote{\footTwoPowers}.

%\subsubsection{powersOf2}
%powersOf2 : set di potenze di 2. Serve perché altrimenti servirebbe un vincolo non lineare su D per limitarne i valori alle sole potenze di 2. \\

\subsubsection{powersOf2}
\powersOfTp{} è la p-esima potenza di 2 dell'insieme P\footnote{\footFreedom}.

 % set di potenze di 2. Serve perché altrimenti servirebbe un vincolo non lineare su D per limitarne i valori alle sole potenze di 2. \\






\newpage




%% F.OBIETTIVO


\subsection{funzione obiettivo}
\iffalse
\begin{math}
 minimize D
\end{math}
\fi

minimizzaLato è la funzione obiettivo, la quale cerca di minimizzare D.




\subsection{vincoli}

\subsubsection{minimaX e minimaY}
%\label{label}
%\hyperref[eq:minimaX]{minimaX}
minimaX si assicura che, per ogni immagine i, D la possa contenere sulle ascisse. \\
minimaY è lo stesso vincolo sulle ordinate.




% @TODO
% \hbox{\pdfliteral{0 0 1 rg}\vrule height1cm width8cm depth0cm\pdfliteral{0 g}}

\newcommand{\Cij}{C\textsubscript{i,j}}

\subsubsection{beforeXU, beforeXL e beforeYU, beforeYL}
beforeXU e beforeXL sono i vincoli che si assicurano che \beforeXij{} abbia il corretto valore rispetto alla differenza tra \Xi{} e la somma di \Xj{} e \Wj{} (ovvero la distanza tra la posizione del primo pixel occupato da i e la posizione dell'ultimo pixel occupato da j). \\
Essi pongono un limite superiore e uno inferiore all'espressione \Xi{}-(\Xj{}+\Wj{}) e, operando in coppia, forzano \beforeXij{} a essere:
\begin{itemize}
	\itemsep0em
	\item 1, se $\Xi{}-(\Xj{}+\Wi{}) \in [0;bigM]$
	\item 0, se $\Xi{}-(\Xj{}+\Wi{}) \in [-bigM;-1]$
\end{itemize}
% differenza tra l'inizio dell'immagine i e  la fine dell'immagine j.
%Essi pongono un limite superiore e uno inferiore all'espressione \Xi-(\Xj+\Wi) ( d'ora in poi indicata con \Cij) e risultano in beforeX i,j=1 se ci,j appartiene a [0,bigM], beforeX i,j=0 se ci,j appartiene a [-bigM, -1].
%differenza tra l'inizio dell'immagine i e  la fine dell'immagine j.
\iffalse
/*
Cx variabile d'aiuto per befX; dicono quanto spazio c'è a “sinistra” tra l'inizio di i e la fine di j
* >= 0 intersezione non possibile, 
* <0 intersezione possibile
Cy simile per befY
*/
\fi
beforeYU e beforeYL sono i corrispettivi vincoli per la variabile \beforeYij{}.


\subsubsection{noIntersezioni}
noIntersezioni si assicura che non ci siano intersezioni, ovvero sovrapposizioni sia sulle proiezioni sulle ascisse che sulle ordinate. Esso limita inferiormente la somma di \beforeXij{} e \beforeYij{} per ogni coppia non ordinata di immagini (i,j) a 1, ovvero ci può essere al massimo una sovrapposizione su un asse tra ascisse e ordinate ma non su entrambe (cioè un'intersezione).



\subsubsection{rotazione}
rotazione si assicura che le immagini possano ruotare solo se il relativo parametro allowRotations è impostato a 1.





\subsubsection{larghezza e altezza}
larghezza imposta la larghezza finale occupata dall'immagine i. Si assicura di tener conto del bleeding e della possibilità che le immagini possano essere ruotate. \\
altezza è un vincolo simile per l'altezza.


\subsubsection{powersOf2U e powersOf2L}
powersOf2U e powersOf2L sono vincoli che definiscono upper e lower bound per D. \\
Il primo si assicura che, in caso D sia maggiore di \powersOfTp{}, \isTp{} sia settato a 0. \\
Il secondo fa la stessa cosa in caso D sia minore di \powersOfTp. 

\subsubsection{TwoPower}
TwoPower si assicura che, in caso usePowerOfTwo sia impostato a 1, D sia una potenza di 2 tra quelle presenti in powersOf2. \\
In caso il parametro sia settato il vincolo causa la somma delle variabili is2 a essere almeno 1\footnote{\footIsTSum}.
Quindi i vincoli powersOf2U e powersOf2L verranno obbligati a portare una variabile \isTp{} a 1 costringendo la differenza tra D e la rispettiva \powersOfTp{} a 0, pena l'infattibilità. \\
Se invece il parametro è a 0, i vincoli powersOf2U e powersOf2L non hanno influenza.



\newpage