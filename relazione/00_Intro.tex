
%	\maketitle
\begin{center}
	\hspace{0pt}
	\vfill
	\Huge{
		\textbf{\thetitle} \\
		\LARGE{\textbf{Progetto: texture atlas}} \\
		\ \newline \newline \newline
		\Large{	Studente: \textbf{\theauthor} } \\
		\normalsize{{Matricola: \textbf{\matNum}} }
	}
	\vfill
	\hspace{0pt}
	%\vspace*{\fill}
	%\pagebreak
\end{center}
\newpage

\tableofcontents    
\newpage


\iffalse



\fi



%\begin{center}
%	\hspace{0pt}
%	\vfill
	\section{Abstract} 
	\ \newline

Il problema trattato è quello delle texture atlas, ovvero immagini di dimensioni notevoli utilizzate come raccoglitori di altre immagini, utili per accelerare l'accesso (spesso da parte della GPU) alle texture utilizzate. \\

Si vuole trovare la minima dimensione D di un'immagine quadrata tale che in un immagine D x D sia possibile inserire tutte le texture date.
	delle texture atlas

%	\vfill
%	\hspace{0pt}
%\end{center}

\newpage